\documentclass{article}
\usepackage[utf8]{inputenc}
\usepackage[spanish]{babel}
\usepackage{listings}
\usepackage{graphicx}
\graphicspath{ {images/} }
\usepackage{cite}
\usepackage{enumerate}

\begin{document}

\begin{titlepage}
    \begin{center}
        \vspace*{1cm}
            
        \Huge
        \textbf{Práctica 1}
            
        \vspace{0.5cm}
        \LARGE
        Descripción del proceso para la movención de dos objetos
            
        \vspace{1.5cm}
            
        \textbf{Juan Manuel Giraldo Botero}
            
        \vfill
            
        \vspace{0.8cm}
            
        \Large
        Despartamento de Ingeniería Electrónica y Telecomunicaciones\\
        Universidad de Antioquia\\
        Medellín\\
        Marzo de 2021
            
    \end{center}
\end{titlepage}

\newpage
\section{Posición inicial}
las dos tarjetas se deben encontrar debajo de la hoja, una encima de la otra.
\section{Posición final}
las tarjetas deben estar encima de la hoja, formando una piramide una recostada en la otra.

\vspace{1cm}

{\huge Pasos}

\begin{enumerate}[1.]
    \item Tomar la hoja desde la esqiuna inferior izquierda.
    \item Mover la hoja hacia la izquierda hasta tener acceso a las trajetas.
    \item Tomar las tarjetas y ponerlas encima de la hoja.
    \item Mover la hoja a la posicion inicial.
    \item Con el dedo pulgar e índice tomar la tarjeta superior.
    \item poner la tarjeta entre los dedos anular y medio.
    \item Con el dedo pulgar e índice tomar restante.
    \item poner la tarjeta restante entre los indice y medio.
    \item Llevar la mano a una posición horizontal.
    \item formar un triangulo con las dos tarjetas teniendo la hoja como base del mismo.
    \item buscar el punto donde las tarjetas se sotienen solas, una contra la otra.
    \item retirar cuidadosamente la mano 

\end{enumerate}


\end{document}
